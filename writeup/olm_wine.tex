\documentclass{llncs}

\usepackage{ifthen,hyperref,subfig,graphicx, color, setspace, lineno} 

%\linenumbers*[1]

%\doublespacing

\begin{document}

\title{Online Labor Markets} 
\author{John J. Horton} 
\institute{Harvard University, Cambridge, MA 02138, USA} 
\maketitle

\begin{abstract} 
In recent years, a number of online labor markets have emerged that
allow workers from around the world to sell their labor to an equally
global pool of buyers. The creators of these markets play the role of
labor market intermediary by providing institutional support and
remedying informational asymmetries. In this paper, I explore
market creators' choices of price structure, price level and
investment in platforms. I also discuss competition among
markets and the business strategies employed by market creators. The paper
concludes with a discussion of the productivity and welfare effects of
online labor.
\end{abstract}

\section{Introduction} 
In the late 1990s, a number researchers began studying the effects
that the Internet was having---or might yet have---on the labor
market.  One question examined was whether we might see the emergence of
entirely online labor markets, where geographically dispersed workers
and employers could make contracts for work sent ``down
a wire.'' Such markets would be an unprecedented
development, as labor markets have always been geographically
segmented. 

Researchers were of mixed opinions: Malone predicted the
emergence of such an ``E-lance'' market \cite{malone1998}, while Autor
was skeptical, arguing that informational asymmetries would make such
markets unlikely \cite{autor2000wlm}. Instead, Autor predicted the
emergence of third-party intermediaries that could use their own
reputation to convey ``high bandwidth'' information about
workers---such as ability, skills, reliability and work ethic---to
buyers who would be unwilling to hire workers based solely on
demographic characteristics and self-reports.

In the approximately 10 years since, we have witnessed the
emergence of a number of truly global online labor markets, as Malone
predicted. By 2009, over 2 million worker accounts had been
created across different markets, with over \$700 million in gross wages paid to workers \cite{frei2009}. However, consistent
with Autor's position, these markets have emerged not ``in the wild,''
but within the context of highly structured platforms created by
for-profit intermediaries.

The ultimate success and trajectory of these markets remains to be
seen. If they become more important, they will raise policy
questions, particularly about labor laws and taxation. They might spur
the already large shift towards part-time employment \cite{oecd2010}
and have implications for inequality and development. 

The purpose of
this paper is to describe the key economic features of online labor
markets. In addition to a positive examination, this paper highlights
features of the markets likely to be relevant for welfare and
productivity. Special attention is given to the ability of these
markets to give workers in developing countries access to buyers in
rich countries.

\section{Overview} 
Online labor markets (OLMs) fall into two broad categories: ``spot''
and ``contest.'' No labor market is truly ``spot'' in the sense of a
commodity market, but certain OLMs feature buyer/seller agreements to trade at agreed prices for certain durations of
time. Examples of spot markets include \href{http://www.oDesk.com}{oDesk},
\href{http://www.Elance.com}{Elance}, iFreelance and
\href{http://www.Guru.com}{Guru}. Workers create online profiles and
buyers post jobs and wait for workers to apply and/or actively
solicit applicants.

In contest markets, buyers propose contests for informational goods
such as logos (e.g., \href{99Designs.com}{99Designs} and
\href{http://www.crowdspring.com/}{CrowdSPRING}), solutions to
engineering problems (e.g.,
\href{http://www.innocentive.com/}{InnoCentive}) and legal research
(e.g., \href{http://www.articleonepartners.com/}{Article One
  Partners}). Workers create their own versions of the good and
the buyer selects a winner from a pool of competitors. In some markets,
the buyer must agree to select (and pay) a winner before they can
post a contest; in other high-stakes markets where a solution may
be unlikely, the buyer is under no obligation to select a winner.

\subsection{Definition} 
Not all people working online do so through markets: some work is
unpaid (e.g., open-source software and Wikipedia) and other work products
are transferred within a firm, such as through conventional
off-shoring. Even within clearly identifiable markets, there is great
diversity. I propose a definition of OLMs that captures the essential
common features of all markets and yet distinguishes the markets from
other examples of online work: a market where (1) labor is exchanged
for money, (2) the product of that labor is delivered ``over a wire''
and (3) the allocation of labor and money is determined by a
collection of buyers and sellers operating within a price system.

\subsection{Nature of labor markets and role for intermediation}
Labor markets are fundamentally different from other kinds of markets
in at least two ways. First, there is no single ``commodity'' of labor
with an immediately observable quality and single prevailing
price---both jobs and workers are idiosyncratic.  This makes it
difficult for firms and workers to find a good match, and even when
matches are formed, it is difficult for either party to know precisely
what they are getting when they enter into contracts. Buyer/seller
information asymmetries, when combined with opportunities for
strategic behavior, can impede markets; if sufficiently severe, they can 
prevent markets from existing
\cite{akerlof1970market}\cite{rothschild1976equilibrium}. Second,
labor is a service that is delivered over time, often accompanied by
relationship-specific investments in human capital (e.g., learning a
particular skill for a particular job), which creates a number of the
incentive issues that make it hard for parties to fully
cooperate \cite{williamson1979}.

In traditional labor markets, third-party intermediaries
such as temp agencies, unions and testing services profit from
supplying information \cite{autor08}. The creators of online labor
markets do the same thing, though their scope is wider and more
comprehensive. They also provide infrastructure like payment and
record-keeping systems, communications infrastructures and search
technology---functions typically provided by a government or by
parties themselves.

\subsection{What the market creators provide}
In order to increase the information on the demand side, OLMs
often offer worker skills tests, manage reputation systems and provide
worker data from prior within-OLM employment, such as hours worked and
wages.  Making buyer feedback public not only prevents adverse
selection, but also serves to reduce moral hazard, as workers make
decisions about effort ``in the shadow'' of the evaluations that they
will likely receive. To increase supply-side information, OLM creators verify buyers' abilities to pay and
reports on their past behavior in the market. For example, oDesk
guarantees that workers will be paid for hourly work, putting the
impetus on the buyer to interrupt an unprofitable relationship.

The influence of the market creator is so pervasive that their role in
the market is closer to that of a government: they determine the space
of permissible actions within market, such as what contractual forms
are allowed and who is allocated decision rights.\footnote{Their
  software even serves a weights-and-measures function traditionally
  performed by governments by keeping universal time for logging
  worker hours.}  Presumably they design their ``institutions'' to
maximize expected profits. For example, they design rules to reduce
the probability of disputes (subject to the constraint imposed by
reducing flexibility). If disputes do arise, the market creators are
likely to be able to settle them quickly using clear rules or
unambiguous assignments of decision rights, such as making buyers the
arbiters of contract compliance.\footnote{Although there are obvious
  drawbacks to such an assignment of rights, it radically reduces the
  space for disputes. This is in fact the precise rule used by
  Amazon Mechanical Turk.}


\section{Price and price structure} 
Market creators have at least three ways to earn revenue: they can
charge membership fees, levy ad valorem charges on payments and charge
buyers and sellers for using the market (e.g., for listing a job,
taking a skills test or applying for a job).\footnote{Typical usage
  fees appear to be modest and may serve as a kind of Pigouvian
  ``tax,'' since some of the activities seem to be over-supplied in
  the markets. The costs of applying for jobs are so low that there is
  a good deal of application ``spam.''} These different structures are
not mutually exclusive and many market creators use a hybrid
structure.

A market creator has to attract both buyers and sellers to a market
and facilitate valuable interactions. There is a growing
literature on ``two-sided'' markets \cite{rochet2004two} that tries to
understand price structure in the presence of membership
externalities. This research focuses on scenarios where the identity
of the party that pays the fees (or receives subsidies) matters. In online labor markets, buyers and sellers
independently arrive at prices after negotiation, strongly suggesting
that the Coase theorem applies, which permits a conventional one-sided
analysis.\footnote{For example, if buyers have to pay a membership
  fee, there will be fewer buyers. This lowers the demand and
  therefore the price of labor, transferring some of the cost of the
  membership fee to sellers.}

Suppose that potential buyer/seller pairs would get value $v$ from
completing a project and would pay a cost $c$, not including any fees,
if the work were intermediated. The outside option is $0$. The market
creator's marginal intermediation costs are assumed to be zero. If an
ad valorem charge $\gamma$ is leveled, the project goes forward if $v
- (1 + \gamma)c > 0$, whereas if a lump sump fee $\tau$ is leveled, $v
- c - \tau > 0$. With the lump sum fee, the buyer/seller pair makes
use of the market so long as the fee is less than the surplus: $\tau <
v - c$. With the ad valorem charge, the pair makes use of the market so
long as $\gamma < 1 - \frac{c}{v}$. In the lump sum case, absolute
surplus matters, whereas in the ad valorem case, project efficiency
matters.

Depending on the distributions of $c$ and $v$, either price structure
might be optimal or some hybrid might be best, but the ad valorem
charge appears to have several practical advantages. First, it
short-circuits the chicken-and-egg dynamics of any platform with a
two-sided nature \cite{caillaud2003chicken}.  No OLM sprang forth
fully formed with contingents of buyers and sellers. To be useful,
the markets needed members; to attract members, they needed to be
useful. An ad valorem charge avoids this problem. Second, setting an
optimal lump sum charge requires knowledge of project surplus, and
surplus could change dramatically as different kinds of work become
more or less popular, or as firms shift more work onto the market. A
firm can change membership fees, but this introduces menu
costs. Finally, groups of buyers can bundle their projects under a
single account and amortize their membership costs over many
transactions, but this strategy offers no benefits when using usage fees.
While membership fees can be important and are used in some markets,
the rest of this analysis focuses on the ad valorem price structure.

\subsection{Setting the optimal ad valorem price level}
Perhaps because of the advantages enumerated above, ad valorem charges
seem to be nearly universally applied, even in the contest
markets. Assume that the buyers are purchasing efficiency units of
labor from homogeneous workers and that there is a single market
clearing price. The market clearing price is $p$ and the quantity of
units bought and sold is $Q$. Figure \ref{fig:av} depicts the problem
in terms of intersecting supply and demand curves determining the
market clearing price and quantity for a given $\gamma$. The market
creator's revenue is indicated by the box with height $p\gamma$ (the
side runs from $p$ to $p(1+\gamma)$) and width $Q_0$. As $\gamma$
grows larger, the quantity is lowered, but the height of the rectangle
increases. The nested revenue box shows that as supply and demand
become more elastic ($S'$ and $D'$), the same size $\gamma$, even if
it leads to the same market clearing price, would lead to a decrease
in the quantity (and hence revenue), which is now at $Q'_0$.

The market creator's profit maximization problem is:
\begin{equation}
\max_\gamma p(\gamma) Q(\gamma) \gamma
\end{equation} 
where $\gamma$ is the ad valorem charge and $p(\gamma)$ and
$Q(\gamma)$ are the resultant prices and quantities in the market. The
profit-maximizing first-order condition is $\left(p'Q + Q'p \right) +
pQ = 0$, which implies that profits are a maximum when
$\epsilon^p_\gamma + \epsilon^Q_\gamma = -1$. The market creator increases the
ad valorem charge until a small change in $\gamma$, say $x\%$ is
offset by a combined $x\%$ decrease in some percentage combination of
market price and quantity. Of course, the quantities
$\epsilon^Q_\gamma$ and $\epsilon^p_\gamma$ are not known and are not
parameters that usually receive attention in economics. However, if we
make some assumptions about the functional form of the supply and
demand curves, we can solve for $\gamma^*$ as a function of the
relevant elasticities.
\begin{figure}[htb]
\begin{center}
 \scalebox{.3}{\input{./figures/sd_t}}
\end{center}
\caption{Market creator profits and market clearing price \label{fig:av}}
\end{figure}
Assume that both the supply and demand curves have constant elasticity
of substitution, $s(p) = Q_sp^\alpha$ and $d(p) = Q_dp^\beta$.  In the
absence of a market creator, assuming the market could function, the efficient
price for labor would be $p_e =e^{\frac{\log Q_d- \log Q_s}{\alpha
    -\beta }}$.  Under intermediation, for the market to clear,
$Q_sp_I^\alpha=Q_d(p_I(1+\gamma))^\beta$. We can solve for the
intermediation market clearing price, $p_I$, and write it in terms of
the efficient market price: $p_I = p_e (\gamma +1)^{\frac{\beta
  }{\alpha -\beta }}$. Because $\gamma > 0$, $\beta < 0$ (downward
sloping demand curve) and $\alpha > 0$ (upward sloping supply curve),
in order for the market to still clear with the ad valorem charge, the price
received by sellers must be lower than in the efficient
market case.\footnote{Note, however that this ``distortion'' from the
  efficient market price is not necessarily an inefficiency, as it is
  the actions of the market creator that make the market possible.} The market creator's profits are:
\begin{equation}
\pi = \left[\underbrace{(Q_sp_e^\alpha) p_e}_{\mbox{\tiny{efficient wage bill}}} \right] \times 
 \left[  \gamma \left(\gamma
   +1\right)^{\frac{\beta (\alpha +1)}{\alpha -\beta}} \right]
\end{equation} 
Solving for the optimal charge, we have: 
\begin{equation}
\gamma^* = \frac{\beta -\alpha }{\alpha  (\beta +1)}
\end{equation} 
If we assume that the supply and demand elasticities have the same
magnitude, i.e., $\alpha = |\beta|$, then in order to give the highest observed ad
valorem charge of 25\% employed by BitWine\footnote{
  \href{http://www.bitwine.com}{BitWine} is a network of freelance
  advisers who charge clients per-minute rates for
  consultations. Advisers are self-styled experts in fields such as
  nutrition, travel, coaching, technology and psychic prediction.},
$\alpha = |\beta| = 9$; in order to give the more standard $\approx
10\%$ used by oDesk, Amazon Mechanical Turk and others, $\alpha =
|\beta| = 21$.  These are remarkably high elasticities. It is not
clear whether constant elasticity of substitution is a reasonable assumption, but if it is, and
assuming that the market creators know their business and are not
radically undercharging, it seems likely that implied elasticities are
large for the simple reason that workers and buyers have ready and
close substitutes for their intermediated transactions: they can make
use of other online labor markets or traditional labor markets, or they
can take their chances and disintermediate.

\section{Competition and specialization} 
It is well beyond the scope of this paper to try to model the market
of intermediation markets, never mind make predictions about the likely
market structure, product types, prices, etc. However, it is possible
to discuss some of the key economic factors and sketch out
areas for future research. The factors likely to affect ultimate
market structure include whether there are economies or diseconomies
of scale in providing intermediation services, barriers to entry and
the potential for product differentiation.

\section{Market creator strategy}
Even after picking a price structure and level, the market creator can
still increase revenues by increasing the size of the wage bill. This can be done by increasing the extent of the market, such as by
recruiting more buyers and sellers, increasing worker productivity or
preventing buyers and sellers from working outside the market.

\subsection{Recruitment that affects supply and demand} 
Let the market creator's initial revenues be $r_0 = p_0 Q_0$. The
market creator is considering changing supply and demand in the market
via recruitment of more buyers and sellers, such as through
advertising.  After the change, the new revenue will be $r_1 = p_1
Q_1$. Define $\Delta x$ as a percentage change in $x$. We can write
$r_1 = p_0Q_0(1+\Delta p)(1+\Delta Q)$. It is worth making the change
from $r_0$ to $r_1$ if
\begin{equation}
\Delta Q + \Delta p + \Delta p \Delta Q > 0
\end{equation} 
We can see that increasing within-market demand unambiguously raises
profits because as $\Delta Q$ increases, so does $\Delta p$, as
positive demand shocks raise both price and quantity. For
supply increases, price and quantity will move in opposite
directions. For small changes in supply or demand, two elasticity
formulas must hold: $\Delta Q = \Delta S + \epsilon^S \Delta p$ and
$\Delta Q = \Delta D + \epsilon^D \Delta p$. Because we are
considering only a supply increase, $\Delta Q = \epsilon^D \Delta p$,
which allows us to re-write the profit-maximizing condition as
$\epsilon^D \Delta p + \Delta p + \Delta p \Delta Q > 0$. Dividing
through by $\Delta p$ (which is negative) and reversing the sign, we
have: $\epsilon^D + \Delta Q < -1$, and since $(\epsilon^D -
\epsilon^S)\Delta p = \Delta S$, the market creator finds it revenue-maximizing
to increase supply so long as:
\begin{equation}
\epsilon_D \left(1 + \frac{\Delta S}{\epsilon^D - \epsilon^S} \right) < -1
\end{equation} 
If supply and demand are highly elastic, $|\epsilon^D - \epsilon^S|$
is large, meaning that small positive changes in supply are likely to
increase revenue. 

\section{Productivity and welfare implications}
Online work offers the cost-saving benefits of telecommuting, such
as reduced congestion and increased flexibility, as well as some advantages
unique to the way such markets appear to be structured. First,
global labor markets permit greater specialization in human
capital. Second, the rapid mixture of workers across and between firms
might speed up innovation spillovers, creating a kind of pseudo
geographic co-location, which has been shown to increase productivity
in other contexts \cite{greenstone2008identifying}. Third, OLMs allow
firms to buy small amounts of labor as needed, lowering the barriers
to entrepreneurship.

OLMs also permit a kind of virtual migration that offers many of the
benefits of physical migration. Assuming that increased virtual labor
mobility will generate effects similar to those of increased real
labor mobility, the potential gains to welfare are enormous
\cite{clemens2008place}. Further, these markets create incentives for
people otherwise disconnected from the global labor market to invest
in their human capital \cite{osterCallCenters2010}.

Given the central role that a country's institutions play in its
economic development \cite{acemoglu-institutions}, it is remarkable how
little these markets demand from the institutions of the worker's home
country. Prospective workers need only to be able to get online and
have some way of receiving remittances. Workers do not need functioning
courts, developed finance sectors, work visas, information about
commodity prices, local reputations or race, class or social
backgrounds required for employment in local labor markets.

\subsection*{Acknowledgments} 
Thanks to the NSF-IGERT Multidisciplinary Program in Inequality \&
Social Policy. Thanks to Robin Yerkes Horton, Richard Zeckhauser, Olga
Rostopshova and Dana Chandler for helpful comments and suggestions.

\bibliographystyle{abbrv}
\bibliography{olm_wine.bib}

\end{document} 



